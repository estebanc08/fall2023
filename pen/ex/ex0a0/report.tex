%  Make this into a pdf document as follows:
%
%
% The edit the Report.tex file appropriately to include only those elements that
% make sense for the assignment you're reporting on.
%
% You can use a tool like TeXShop or Texmaker or some other graphical tool
% to convert Report.text into a pdf file.
%
% If you are making this with command line tools, you'd run the following command:
%
%     latex Report.tex
%
% That will generate a dvi (device independent) document file called Report.dvi
% The pages reported in the table of contents won't be correct, since latex only
% processes one pass over the document. To adjust the page numbers in the contents,
% run latex again:
%
%    latex Report.tex
%
% Then run
%
%   dvipdf Report.dvi
%
% to generate Report.pdf
%
% You can view this file to check it out by running
%
% xdg-open Report.pdf
%
% That's it.
  
\def\cvss(#1,#2,#3,#4,#5,#6,#7,#8,#9){
	\indent\textbf{CVSS Base Severity Rating: #9}  AV:#1 AC:#2 PR:#3 UI:#4 S:#5 C:#6 I:#7 A:#8}
  
\def\ttp(#1, #2, #3, #4, #5, #6){
   \indent\textbf{#1:} #2 \\
   \indent\indent\textbf{#3:} #4 \\
   \indent\indent\indent\textbf{#5:} #6 \\}

\documentclass[notitlepage]{article}

\usepackage{bibunits}
\usepackage{comment}
\usepackage{graphicx}
\usepackage{amsmath}
\usepackage{datetime}
\usepackage{numprint}

% processes above options
\usepackage{palatino}  %OR newcent ncntrsbk helvet times palatino
\usepackage{url}
\usepackage{footmisc}
\usepackage{endnotes}

\setcounter{secnumdepth}{3}
\begin{document}

\nplpadding{2}
\newdateformat{isodate}{
  \THEYEAR -\numprint{\THEMONTH}-\numprint{\THEDAY}}
  
\title{Penetration Test - Exercise 0a0}
\author{Esteban Calvo}
\date{\isodate\today}

\maketitle

\tableofcontents

\newpage
\section{Technical Report}

  \subsection{Finding: \emph{Network User Compromised}}
	\subsubsection*{Severity Rating}
	Risk Factor: High
    \cvss(N,L,N,N,C,L,L,L,8.3)
		
  	\subsubsection*{Vulnerability Description}
  		A list of hashes was found using a mimikatz exploit. This list was used against
        a list of known hashes to find the credentials of a network user. This attack revealed
        the credentials of one user. 

  	\subsubsection*{Confirmation method}
  	The hashes that had been previously found must be formatted in \textbf{user:hash} format. The use
    of the John the Reaper tool along with the rockyou wordlist must be employed as follows
    \begin{verbatim}
john --wordlist=<pathToRockYou.txt> --format=NT passwordHashes.txt
    \end{verbatim}
    
    \subsubsection*{Mitigation or Resolution Strategy}
    It is recommended to change all user passwords and to make sure that the hashes are checked against
    well known wordlists such as rockyou and that more password requirements are enforced such as longer password
    lengths as is recommended by NIST Publication 800-63B. Creating more complex passwords checked against
    dictionaries and longer passwords must be employed to prevent this kind of attack from occurring again.



\section{Attack Narrative}
    \subsection{Formatting}
    The lsadump with the local Security Account Manager (SAM) NT hashes was used for this exercise. The specifications
    for the John the Ripper tool specified that it must be in \textbf{user:hash} format and therefore the file
    was formatted as follows for all users
    \begin{verbatim}
Administrator:d9a8...dab9
    \end{verbatim}

    \subsection{Password Cracking}
    As stated above, John the Ripper was used as follows 
    \begin{verbatim}
john --wordlist=<pathToRockYou.txt> --format=NT passwordHashes.txt
    \end{verbatim}
    which revealed the following partially hidden password and login
    \begin{verbatim}
User: d.darkblood
Password: De...09
    \end{verbatim}

    \subsection{MITRE ATT{\&}CK Framework TTPs}
   
   \subsubsection*{}
   \ttp(TA0006, Credential Access, T1110, Brute Force, .002, Password Cracking)
   
   \end{document} 

