%  Make this into a pdf document as follows:
%
%
% The edit the Report.tex file appropriately to include only those elements that
% make sense for the assignment you're reporting on.
%
% You can use a tool like TeXShop or Texmaker or some other graphical tool
% to convert Report.text into a pdf file.
%
% If you are making this with command line tools, you'd run the following command:
%
%     latex Report.tex
%
% That will generate a dvi (device independent) document file called Report.dvi
% The pages reported in the table of contents won't be correct, since latex only
% processes one pass over the document. To adjust the page numbers in the contents,
% run latex again:
%
%    latex Report.tex
%
% Then run
%
%   dvipdf Report.dvi
%
% to generate Report.pdf
%
% You can view this file to check it out by running
%
% xdg-open Report.pdf
%
% That's it.
  
\def\cvss(#1,#2,#3,#4,#5,#6,#7,#8,#9){
	\indent\textbf{CVSS Base Severity Rating: #9}  AV:#1 AC:#2 PR:#3 UI:#4 S:#5 C:#6 I:#7 A:#8}
  
\def\ttp(#1, #2, #3, #4, #5, #6){
   \indent\textbf{#1:} #2 \\
   \indent\indent\textbf{#3:} #4 \\
   \indent\indent\indent\textbf{#5:} #6 \\}

\documentclass[notitlepage]{article}

\usepackage{bibunits}
\usepackage{comment}
\usepackage{graphicx}
\usepackage{amsmath}
\usepackage{datetime}
\usepackage{numprint}

% processes above options
\usepackage{palatino}  %OR newcent ncntrsbk helvet times palatino
\usepackage{url}
\usepackage{footmisc}
\usepackage{endnotes}

\setcounter{secnumdepth}{3}
\begin{document}

\nplpadding{2}
\newdateformat{isodate}{
  \THEYEAR -\numprint{\THEMONTH}-\numprint{\THEDAY}}
  
\title{Exercise 05}
\author{Esteban Calvo}
\date{\isodate\today}

\maketitle

\tableofcontents

\newpage


\section{Attack Narrative}

\subsection{TCP Scan}
    To run an nmap scan on arts tailor shop, the following command was performed
    \begin{verbatim}
        sudo nmap -sV -O www.artstailor.com
    \end{verbatim}
    and this yielded the following results:
    \begin{verbatim}
        Starting Nmap 7.94 ( https://nmap.org ) at 2023-09-20 22:14 EDT
        Nmap scan report for www.artstailor.com (172.70.184.133)
        Host is up (0.00058s latency).
        rDNS record for 172.70.184.133: ns.artstailor.com
        Not shown: 997 closed tcp ports (reset)
        PORT   STATE SERVICE VERSION
        22/tcp open  ssh     OpenSSH 9.2p1 Debian 2 (protocol 2.0)
        53/tcp open  domain  ISC BIND 9.18.16-1~deb12u1 (Debian Linux)
        80/tcp open  http    Apache httpd 2.4.57 ((Debian))
        No exact OS matches for host (If you know what OS is running on it, see https://nmap.org/submit/ ).
        TCP/IP fingerprint:
        OS:SCAN(V=7.94%E=4%D=9/20%OT=22%CT=1%CU=43456%PV=N%DS=2%DC=I%G=Y%TM=650BA70
        OS:A%P=x86_64-pc-linux-gnu)SEQ(SP=108%GCD=1%ISR=10A%TI=Z%II=I%TS=A)OPS(O1=M
        OS:5B4ST11NW7%O2=M5B4ST11NW7%O3=M5B4NNT11NW7%O4=M5B4ST11NW7%O5=M5B4ST11NW7%
        OS:O6=M5B4ST11)WIN(W1=FE88%W2=FE88%W3=FE88%W4=FE88%W5=FE88%W6=FE88)ECN(R=Y%
        OS:DF=Y%T=40%W=FAF0%O=M5B4NNSNW7%CC=Y%Q=)T1(R=Y%DF=Y%T=40%S=O%A=S+%F=AS%RD=
        OS:0%Q=)T2(R=N)T3(R=N)T4(R=N)T5(R=Y%DF=Y%T=40%W=0%S=Z%A=S+%F=AR%O=%RD=0%Q=)
        OS:T6(R=N)T7(R=N)U1(R=Y%DF=N%T=40%IPL=164%UN=0%RIPL=G%RID=G%RIPCK=G%RUCK=G%
        OS:RUD=G)IE(R=Y%DFI=N%T=40%CD=S)

        Network Distance: 2 hops
        Service Info: OS: Linux; CPE: cpe:/o:linux:linux_kernel
        
        Nmap done: 1 IP address (1 host up) scanned in 25.18 seconds
    \end{verbatim}
    From this command, we can several open ports as well as their operating system. These ports are
    pretty standard ports and there was not much to observere here.

\subsection{UDP Scan}
    To run nmap on the UDP ports, we used the command
    \begin{verbatim}
        sudo nmap -sU -p 1-256 -sV www.artstailor.com
    \end{verbatim}
    and got the following results:
     \begin{verbatim}
        Starting Nmap 7.94 ( https://nmap.org ) at 2023-09-20 22:35 EDT
        Nmap scan report for www.artstailor.com (172.70.184.133)
        Host is up (0.00066s latency).
        rDNS record for 172.70.184.133: ns.artstailor.com
        Not shown: 254 closed udp ports (port-unreach)
        PORT   STATE         SERVICE VERSION
        40/udp open|filtered unknown
        53/udp open          domain  ISC BIND 9.18.16-1~deb12u1 (Debian Linux)
        Service Info: OS: Linux; CPE: cpe:/o:linux:linux_kernel

        Service detection performed. Please report any incorrect results at https://nmap.org/submit/ .
        Nmap done: 1 IP address (1 host up) scanned in 369.53 seconds    
     \end{verbatim}
    Here, the interesting port was port 40 which may be a threat if it is running tcp, but there was little
    to suggest that this port was a threat in the current state.

\subsection{Comparison}
    From the scan, we can clearly see that TCP is much faster than UDP considering that tcp scanned
    1000 ports and ran in 25 seconds while UDP scanned 256 ports and took 370 seconds. The reason
    for this lies in the protocol itself as a three-way handshake is much easier to establish
    using TCP and UDP's stateless protocol.

\subsection{Searchsploit}
    Using Searchsploit on all the ports did not reveal any potential vulnerabilites. It is possible
    There previously some vulnerabilities that have been patched with recent updates, but 
    there is no reason to belive that the current setup has any real risks.

\subsection{Key}
    To find the key, wireshark was used to examine the packets sent to the suspicous UDP port 40.
    Here, some of the packets sent back and forth contained the key which turned out to combine to:
    \begin{verbatim}
        KEY007-52kyxvjHNa8SNF/s55JH0A
    \end{verbatim}
	

\end{document} 
