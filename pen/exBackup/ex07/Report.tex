%  Make this into a pdf document as follows:
%
%
% The edit the Report.tex file appropriately to include only those elements that
% make sense for the assignment you're reporting on.
%
% You can use a tool like TeXShop or Texmaker or some other graphical tool
% to convert Report.text into a pdf file.
%
% If you are making this with command line tools, you'd run the following command:
%
%     latex Report.tex
%
% That will generate a dvi (device independent) document file called Report.dvi
% The pages reported in the table of contents won't be correct, since latex only
% processes one pass over the document. To adjust the page numbers in the contents,
% run latex again:
%
%    latex Report.tex
%
% Then run
%
%   dvipdf Report.dvi
%
% to generate Report.pdf
%
% You can view this file to check it out by running
%
% xdg-open Report.pdf
%
% That's it.
  
\def\cvss(#1,#2,#3,#4,#5,#6,#7,#8,#9){
	\indent\textbf{CVSS Base Severity Rating: #9}  AV:#1 AC:#2 PR:#3 UI:#4 S:#5 C:#6 I:#7 A:#8}
  
\def\ttp(#1, #2, #3, #4, #5, #6){
   \indent\textbf{#1:} #2 \\
   \indent\indent\textbf{#3:} #4 \\
   \indent\indent\indent\textbf{#5:} #6 \\}

\documentclass[notitlepage]{article}

\usepackage{bibunits}
\usepackage{comment}
\usepackage{graphicx}
\usepackage{amsmath}
\usepackage{datetime}
\usepackage{numprint}

% processes above options
\usepackage{palatino}  %OR newcent ncntrsbk helvet times palatino
\usepackage{url}
\usepackage{footmisc}
\usepackage{endnotes}

\setcounter{secnumdepth}{3}
\begin{document}

\nplpadding{2}
\newdateformat{isodate}{
  \THEYEAR -\numprint{\THEMONTH}-\numprint{\THEDAY}}
  
\title{Penetration Test Report Title}
\author{Esteban Calvo}
\date{\isodate\today}

\maketitle

\tableofcontents

\newpage
\section{Executive Summary}



\subsection{Project Overview}
The first words of your report should \emph{always} be the client's company name.
You want them to know that you know who you're working for!

Use active voice!
``The Cozy Croissant LLC retained the services of Pr0b3 Security to test the security of their network to ensure that ....''

Here's how to include a picture:

\includegraphics[width=4in]{foo.png}

This is the executive summary.
It should should definitely be included in your final report but should be omitted in interim reports.


\subsection{Goals}

This reiterates what the client hoped to get out of the penetration
test.

\subsection{Risk Ranking/Profile}

This is an overall summary of the risk that the client suffers from. This effectively indicates the general security posture of the company as determined by your penetration testing.

\subsection{Summary of Findings}

This should be a brief executive-level description of the most important findings you uncovered.


\subsection{Recommendation Summary}

This should be a high-level discussion of your advice to remediate any problems that were uncovered.
This should be strategic in nature, not a blow-by-blow tactical summary.
It should get at the major themes and issues confronting the client.


\section{Technical Report}

Feel free to include an introduction if it suits your communication style.
You may omit it if you prefer to do so.

% Include one of these headings for each finding.

  \subsection{Finding: \emph{Descriptive Name}}
  
    These findings are for the customer.
    Never mention the class in these findings.
    Never refer to the attack narrative in these findings.
    You will want to be able to copy-and-paste these into a more complete report.

	\subsubsection*{Severity Rating}
		Here you identify the severity (using the CVSS base score metrics)
		and point out the potential outcome of exploitation of this
		vulnerability.

		Include the CVSS base score designation \textbf{including its
		subcomponent values} to document your	severity rating.\\
		Example: \\
	   	This shows you how to format a CVSS score for your report.
	   	(You should be able to look at CVSS documentation to
	   	determine what the parameters A, H, L, N, U, H, L, N,
	   	<number> should represent.)
	   	
	    
		\cvss(A,H,L,N,U,H,L,N,5.4)
		
  	\subsubsection*{Vulnerability Description}
  		Here you provide a brief description of the nature of the vulnerability
  		including where the vulnerability is present (what machine and
  		what service).
  		
  	\subsubsection*{Confirmation method}
  	
		This section contains the information necessary for the
		client to verify that the vulnerability still exists.
		(Note: inability to confirm that the vulnerability
		does not exist using this method does not
		guarantee that the vulnerability has been addressed
		or mitigated.)
		
		The best confirmation method sections contain a few commands
		to execute. The vulnerability is confirmed by comparing
		the result to the expected result on a vulnerable
		host/network. The confirmation method should be simple
		and is usually \emph{not} exactly the same as what you
		did to discover and exploit the vulnerability.
		This is something the client's admins, with full
		highest privilege access can do to confirm the
		vulnerability is present.
		
    \subsubsection*{Mitigation or Resolution Strategy}
    
    	This is where you describe how to address the problem.
    	Can it be completely solved or can you, at least, reduce the
    	likelihood that the vulnerability can be exploited?
		



\section{Attack Narrative}

	This section is included only in interim reports,
	not the final report.

	
	You should provide enough information for a knowledgeable
	penetration tester to reproduce your results.

	If active intelligence was gathered, you provide enough details
	of the methods employed so that a knowledgeable penetration
	tester could gather the same information.
	
	If a vulnerability was exploited, you provide enough details
	of your activities so that a knowledgeable penetration tester
	could reproduce your results under the ame circumstances.

    \subsection{MITRE ATT{\&}CK Framework TTPs}
    
    In this subsection of the Attack Narrative, you provide the
    MITRE ATT{\&}CK Framework TTPs that were employed in the
    activity.
    
    This example shows what you write in your LaTeX source
    in order to generate a TTP string that we will use for
    grading purposes. Do \textbf{NOT} put two occurrences of
    the TTP in your reports!
    
    \subsubsection*{Examples:}
    
	
	\subsubsection*{This shows how you can format a MITRE ATT{\&}CK framework TTP for printing in your document:}
	\ttp(TA0043, Reconnaissance, T1593, Search Open Websites/Domains, .002, Search Engine)
    
	\subsubsection*{Here's another TTP Example:}
	\ttp(TA0002, Execution, T1053, Scheduled Task/Job, .003, Cron) 
\end{document} 
